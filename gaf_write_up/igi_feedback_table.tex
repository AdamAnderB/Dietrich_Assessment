
\begin{longtable}{|p{0.15\textwidth}|p{0.80\textwidth}|p{0.05\textwidth}|}
\caption{Descriptive Feedback and Scores for IGI Learning Outcomes} \\
\hline
\textbf{Learning Outcome} & \textbf{Descriptive Feedback} & \textbf{Score} \\
\hline
\endfirsthead
\hline
\textbf{Learning Outcome} & \textbf{Descriptive Feedback} & \textbf{Score} \\
\hline
\endhead
IGI1.1 & \#\#\# **Response to Submission**  Your research proposal on the beginner Chinese curriculum provides a comprehensive and thought-provoking exploration of the interplay between language learning and cultural contexts intrinsic to Chinese languaculture. Here's an evaluation of your work utilizing the rubric provided, along with suggestions for improvement:  1. **Research Questions**:     - Your research questions are clear and focused on the specific relationship between vocabulary, curriculum design, and cultural context. This clarity will aid in directing your study effectively.  2. **Rationale/Statement of Importance**:     - The rationale articulates well why understanding the languaculture behind language education is crucial. By linking linguistic fluency to cultural understanding, you underscore the practical implications of your research, which is excellent. However, to enhance this section, you might want to reflect more on how your personal experiences with language learning or teaching—particularly your own encounters with Chinese languaculture—have shaped your perspective on this issue.  3. **Proposed Data Collection Methods**:     - The methods outlined for data collection show a reasonable balance between quantitative and qualitative approaches. Conducting interviews with educators will undoubtedly enrich your findings. To deepen this, consider including reflections on how your own learning experiences influenced the questions you formulated for the interviews—this could align better with your personal worldview.  4. **Proposed Analytic Procedures**:     - Your proposed analysis seems robust, particularly in integrating qualitative insights from interviews with quantitative data from textbooks. However, it would benefit from a personal angle. How have your experiences as a language learner or educator shaped your understanding of what should be included in a beginner curriculum? Including your worldview here can give more context to your analytical framework.  5. **Statement of Possible Implications/Outcomes for Research and/or Practice**:     - This section summarizes potential contributions clearly. Emphasizing the real-world applications of your findings, such as curricular development and teaching methodologies, shows how your project is not solely theoretical. Highlighting how cultural contexts from your personal experience inform these implications would enhance your connection to the research outcomes.  \#\#\# **Personal Worldview Reflection** To align your proposal more closely with the rubric, I encourage you to incorporate specific narratives or experiences that have shaped your understanding of language learning amidst cultural contexts. For instance, reflecting on any challenges you faced when learning Mandarin could enrich the proposal. Did you notice discrepancies between what you learned and colloquial usage? Did interactions with native speakers provide insights into the languaculture that formal education overlooked? By integrating such reflections, you would not only clarify your personal worldview but also establish a richer context for your research.  \#\#\# **Scoring Level Recommendation** Given this evaluation, I would place your proposal at Level 2.0 in the rubric because while you present an interesting and relevant study, the personal worldview component is somewhat underdeveloped. By weaving in your personal narrative and cultural influences more explicitly, you can elevate your score to Level 3.0 or higher, creating a more engaging and comprehensive proposal.  **Overall, you have a strong foundation. With these enhancements, your project can reflect not only the academic rigour of your research but also the personal insights that truly make learning a comprehensive journey.** & 2 \\
\hline
IGI1.2 & Your research proposal on the "Beginner Chinese Curriculum" demonstrates a thoughtful approach to the intersection of language and culture. Below is a reflective response based on the rubric provided:  \#\#\# Reflection on Personal Worldview  Through this research proposal, I recognize that my personal worldview has been shaped significantly by my experiences with language learning and cultural immersion. Growing up in a multicultural environment, I was always aware of the nuances in language that extend far beyond vocabulary, including the cultural contexts that inform how language is used in everyday life. My experiences have led me to value not just words, but the meanings, histories, and sentiments that accompany them, which seem integral to achieving fluency.   \#\#\#\# Acknowledgment of Cultural Contexts  The proposal reflects the belief that language learning is incomplete without understanding the cultural underpinnings—what I would refer to as "languaculture." This perspective stems from my own challenges in learning languages that often do not directly translate phrases or concepts due to cultural differences. For instance, early on in my language studies, I struggled with idiomatic expressions in Mandarin that lacked direct English counterparts. This highlighted for me the necessity of integrating cultural education alongside linguistic skill development.  Moreover, working with diverse groups of learners, I have observed how cultural contexts impact motivations, learning styles, and ultimately, language acquisition. My interactions with educators and students from various backgrounds have confirmed that successful language education must consider the cultural values and practical applications within daily conversations. In proposing an analysis of vocabulary sets in beginner curriculums, I aim to advocate for an approach that closely aligns with how native speakers truly converse, thereby bridging the gap between language education and cultural competency.  \#\#\#\# Synthesizing Insights and Acknowledging Limitations  In synthesizing these insights, I aim to highlight the disconnection between educational materials and colloquial language use, as articulated in the proposal. This gap has real implications for language learners who, like me, seek to use their skills in meaningful ways. At the same time, I acknowledge the limitations of my perspective. Different learners may prioritize different aspects of language learning—some may find value in structured vocabulary and traditional curriculums, while others may thrive in immersive, culturally rich environments.   Ultimately, this nuanced understanding of my personal worldview serves as a foundation for my research. It drives my inquiry into how beginner Chinese curriculums can evolve to reflect the realities of Chinese languaculture, fostering better communication and cultural understanding for learners at all levels.  \#\#\# Score Assessment  Given the depth of reflection and the acknowledgment of cultural contexts in shaping my personal worldview, I believe this response aligns with **Level 4.0** on the rubric. It comprehensively explains my personal experiences and insights while integrating multiple cultural contexts and acknowledging potential limitations, ensuring a well-rounded perspective on the subject matter.   ---  By revisiting this proposal with clearer reflections of how my cultural background has influenced my approach, I hope to create a richer dialogue on the importance of integrating culture into language education. If there are specific areas where you would like me to expand further or if you have additional questions on this topic, please let me know! & 4 \\
\hline
IGI2.1 & \#\#\# Response to Beginner Chinese Curriculum Research Proposal  Your proposal on designing a study to investigate beginner Chinese language curriculums through the lens of languaculture is incredibly well-structured and insightful. Below, I will address specific elements of your project and provide suggestions for enhancing clarity and depth.  1. **Research Questions**: Your questions are relevant and clearly articulated. They establish a foundation for exploring the intersection of language and culture within the context of teaching Mandarin. It might be beneficial to expand on how you will define "value" in terms of educators’ perspectives on vocabulary. This could help target more specific cultural practices or beliefs influencing curriculum choices.  2. **Rationale/Statement of Importance**: You effectively highlight the gap in current curriculums and the importance of culturally relevant vocabulary for language learners. To strengthen this, consider including personal insights or any specific experiences that led you to this topic. For instance, if you encountered difficulties when learning Mandarin due to cultural discrepancies in vocabulary, sharing such experiences could enhance your argument and provide a personal connection to the research.  3. **Proposed Data Collection Methods**: Your methods are both comprehensive and practical. In addition to analyzing textbooks and conducting interviews, you might consider incorporating participant observations in language classrooms or language exchange groups. This could provide observational data on how vocabulary is utilized by learners in real-time situations, adding another dimension to your analysis.  4. **Proposed Analytic Procedures**: The combination of quantitative and qualitative analysis is a well-thought-out approach. It might be helpful to clarify how you will display your findings. For example, will you use tables, charts, or narrative descriptions? Additionally, discussing potential coding strategies for qualitative data could demonstrate a deeper understanding of how to analyze interview responses related to languaculture.  5. **Possible Implications/Outcomes**: You anticipate practical applications for your research, which is essential for its potential impact. It may be useful to discuss who specifically could benefit from your findings—such as educators, curriculum developers, or even students. Acknowledging any limitations of your study (such as the variability in learners' cultural backgrounds or differing regional usages of Mandarin) could also provide more nuance to your conclusions.  \#\#\# Personal Reflection and Cultural Context  In reflecting on your proposal, it’s evident that your approach stems from a culturally conscious perspective, highlighting the significance of languaculture in language education. To achieve a higher score on the rubric, consider synthesizing multiple cultural contexts that have shaped your own understanding of language learning. For example, how has your background, experiences in language acquisition, or interactions with native speakers influenced your views on curriculum design? Acknowledging these aspects could provide a more comprehensive explanation of your personal worldview while connecting to broader themes in language education.  In summary, your proposal is already strong, and incorporating personal elements along with enhanced clarity in your methods and analyses could further strengthen it. I look forward to seeing how you develop these ideas in your final submission. & 2 \\
\hline
IGI2.2 & Thank you for sharing your research proposal on the Beginner Chinese Curriculum and how it reflects Chinese languaculture. Here’s a thoughtful response:  \#\#\# **Reflection on Proposal**  Your proposal presents a significant and timely investigation that addresses a real gap in current language education methodologies, particularly for learners of Mandarin. Your research questions focus on how Chinese language educators value different types of beginner vocabulary sets and their impact on understanding the broader concepts of Chinese languaculture. This is particularly relevant given the increasing global interest in Mandarin as a second language.  \#\#\# **Cultural Contexts and Personal Worldview**  1. **Understanding Personal Connection**:    It's apparent that your interest in this topic might stem from your personal experiences with language learning or exposure to Chinese culture. Reflecting on your own journey in learning Mandarin, whether it was through formal education or interaction with native speakers, can further enrich your research. For instance, how did your own understanding of Chinese culture influence your approach to learning the language? Were there specific cultural elements or societal nuances you encountered that shaped your perception and fluency in Mandarin?  2. **Acknowledging Cultural Contexts**:    The proposal hints at personal stakes in the methodology of teaching language to non-native speakers. By exploring how various vocabulary sets are used in different contexts, you're implicitly recognizing the influence of cultural norms on language learning. This points to an understanding that language is not simply a set of grammatical rules but is deeply intertwined with cultural practices and values. Sharing specific examples from your own language-learning experiences can enhance this acknowledgment. For instance, were there moments in your learning when colloquial phrases differed greatly from textbook vocabulary, and how did that impact your fluency or understanding of everyday interactions?  3. **Synthesis of Cultural Influences**:    To reach the highest scoring level, consider delving deeper into how multiple cultural contexts inform your view of education, language acquisition, and communication. This could include the influence of Western educational philosophies versus Eastern ones, and how these might impact curriculum design. Additionally, reflecting on societal attitudes toward language learning in different settings can provide a broader perspective. Were your learning experiences informed by cultural narratives that typically favor certain types of communication over others?  4. **Limitations and Future Directions**:    Acknowledging the limitations of your own experiences and perspectives can offer a balanced view. What assumptions might you bring to your research based on your cultural background? How do these potential biases inform your understanding of what constitutes 'effective' vocabulary in language learning? By addressing these aspects, you demonstrate a thoughtful awareness of the complexity of cultural influences on both education and personal identity.  \#\#\# **Conclusion**  Overall, your research proposal demonstrates a clear and relevant focus on improving Chinese language education through an understanding of culturals nuances. However, enriching your analysis with reflections on your personal experiences and addressing the interrelatedness of multiple cultural contexts will create a more comprehensive narrative. By doing so, you'll not only strengthen your proposal but also enhance your understanding of how these cultural contexts shape both curriculum and personal identity in language learning.  In summary, your exploration of languaculture, specifically in the context of Mandarin language education, has the potential to bridge gaps in understanding and provide practical benefits for future learners. By being reflective about your own viewpoint and acknowledging broader cultural influences, you can ensure a nuanced approach to your research.  \#\#\# Questions for Further Thought: - How did your background inform your initial perspectives on learning Mandarin? - What experiences did you have with cultural misunderstandings while learning the language? How did those shape your understanding of language as a social tool? - In what ways do you think language education can better incorporate cultural elements that resonate with learners' personal experiences?  This reflective framework can be beneficial as you continue to develop your research proposal. Good luck with your project, and I look forward to seeing where you take it! & 3 \\
\hline
IGI3.1 & \#\#\# Response to Languaculture Research Proposal  Your proposal on investigating the relationship between beginner Chinese vocabulary education and Chinese languaculture is comprehensive and insightful. Here are my reflections and considerations based on the rubric provided.  1. **Research Questions**: Your research questions are well-defined and relevant to the interplay of language and culture. By examining how educators value vocabulary and how this reflects larger cultural ideologies, you position your research within a significant area of inquiry.   2. **Rationale/Statement of Importance**: You effectively articulate the significance of your study, discussing the gap in current curriculums related to cultural context and fluency. However, to enhance your rationale, you might consider including a personal anecdote about your own experiences learning Chinese or observing these cultural discrepancies. This could provide a more robust connection to your personal worldview and highlight the importance of your research on a more personal level.  3. **Data Collection Methods**: Your proposed methods for data collection are thorough and well-structured. The comparative analysis of educational texts and native literature will provide a rich basis for understanding the discrepancies in vocabulary. Additionally, the interviews with curriculum creators are a strong method to gain qualitative insights. It might be useful to include how you plan to recruit these educators for interviews and ensure a diverse representation of perspectives.  4. **Analytic Procedures**: You clearly outline a mixed-methods approach that combines quantitative and qualitative analysis, which is essential for understanding the complexities of languaculture. As you plan your analysis, consider discussing potential limitations in your data, such as the bias in selecting materials or the linguistic variations in different regions of China, which could influence vocabulary use.  5. **Implications/Outcomes**: Your proposed outcomes center around improving language education and understanding the values embedded in Chinese culture, which is commendable. To strengthen this section, you might explore how the findings could affect specific groups (e.g., educators, curriculum developers, students) and the broader implications for intercultural communication.  **Personal Worldview Reflection**: In considering the scoring criteria, I see an opportunity for you to enhance your response to Level 4. To align with this scoring, you could synthesize multiple cultural contexts that have influenced your understanding of language learning, such as your background, the culture of your language instructors, and the significant cultural elements of China itself. Acknowledging how these contexts shape your views on effective language education and fluency could enrich your proposal and connect your research more personally.  Overall, your proposal is compelling and thoughtfully designed, contributing to the understanding of languaculture in language education. Further personal reflections linked to cultural contexts could elevate the depth of your discussion and align more closely with the expectations set by the rubric. I encourage you to incorporate these suggestions as you refine your proposal. & 2 \\
\hline
\end{longtable}